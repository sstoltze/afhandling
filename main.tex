\documentclass[a4paper,openbib,10pt]{memoir}
\usepackage[utf8]{inputenc}
\usepackage[english]{babel}
\usepackage[T1]{fontenc}
\usepackage{lmodern}
\usepackage{amsmath}
\usepackage{amssymb}
\usepackage{amsthm}
\usepackage{mathtools}
\usepackage{listings}
\usepackage{enumerate}
\usepackage{enumitem}
\usepackage[dvips]{graphicx}
\usepackage{wrapfig}

\usepackage{csquotes}
\usepackage[backend=bibtex,style=alphabetic]{biblatex}
\addbibresource{phdbooks.bib}

\usepackage[english, status=draft]{fixme}

\usepackage[all,cmtip]{xy}
\entrymodifiers={+!!<0pt,\fontdimen22\textfont2>}

\setlrmarginsandblock{*}{4cm}{*}
\checkandfixthelayout

\hyphenation{to-po-lo-gi-cal}

\theoremstyle{plain}
\newtheorem{theorem}{Theorem}
\newtheorem{lemma}[theorem]{Lemma}
\newtheorem{proposition}[theorem]{Proposition}
\newtheorem{corollary}[theorem]{Corollary}
\newtheorem*{conjecture}{Conjecture}

\theoremstyle{definition}
\newtheorem{definition}[theorem]{Definition}
\newtheorem{example}[theorem]{Example}

\theoremstyle{remark}
\newtheorem{remark}[theorem]{Remark}

\newcommand{\ip}[2]{\left< #1, #2 \right>}
\newcommand{\size}[1]{\left| #1 \right|}

\let\O\undefined

\newcommand{\N}{\mathbb{N}}
\newcommand{\Z}{\mathbb{Z}}
\newcommand{\Q}{\mathbb{Q}}
\newcommand{\R}{\mathbb{R}}
\newcommand{\C}{\mathbb{C}}
\newcommand{\A}{\mathcal{A}}
\newcommand{\delim}{\:}
\newcommand{\X}{X_{m,n}}
\newcommand{\sdel}{\delim\Bigg\vert\delim}
\newcommand{\SUT}[1]{\SU_{#1}/T^{#1-1}}
\newcommand{\GLB}[1]{\GL_{#1}/B_{#1}}
\newcommand{\UT}[1]{\U_{#1}/T^{#1}}
\newcommand{\GL}{\mathrm{GL}}
\newcommand{\U}{\mathrm{U}}
\newcommand{\SU}{\mathrm{SU}}
\newcommand{\SL}{\mathrm{SL}}
\DeclareMathOperator{\O}{O}
\DeclareMathOperator{\SO}{SO}
\DeclareMathOperator{\GS}{GS}
\DeclareMathOperator{\Ext}{Ext}
\DeclareMathOperator{\Hom}{Hom}
\DeclareMathOperator{\im}{im}
\DeclareMathOperator{\Tr}{Tr}
\DeclareMathOperator{\Id}{Id}
\DeclareMathOperator{\supp}{supp}
\DeclareMathOperator{\pr}{pr}
\DeclareMathOperator{\spa}{span}
\DeclareMathOperator{\inv}{inv}
\DeclareMathOperator{\Det}{Det}

\newcommand{\closure}[2][3]{{}\mkern#1mu\overline{\mkern-#1mu#2}}

\newcommand{\set}[1]{\left\{ #1 \right\}}
\newcommand{\com}[1]{\left[#1,#1\right]}

\newcommand{\lie}[1]{\left[ #1 \right]}

\renewcommand{\a}{\alpha}
\renewcommand{\b}{\beta}
\newcommand{\g}{\gamma}
\newcommand{\e}{\varepsilon}
\renewcommand{\d}{\delta}
\renewcommand{\l}{\lambda}

\begin{document}

\frontmatter

\include{front}
\include{abstract} % Skriv om
\tableofcontents
\include{introduktion} % Skriv om
\mainmatter


\chapter{The space $\X$}

% Snak snak. M�ske deles op i to?
% Noget om starten, hvorfor dette er interessant. Se bagside af blok
% Ogs� noget om rummene, definition, symmetrier (specielt X_{m,n}
% \cong X_{n,m}, m�ske filtreringen?
This chapter will introduce the spaces to be studied, along with a
brief listing of various useful structure.

\begin{definition}
  For natural numbers $m$ and $n$, the space $X_{m,n} \subset
  \C^{mn}$ is defined as
  \[ X_{m,n} = \set{(a_1,\dots,a_n) \in (\C^m)^n \delim\Bigg\vert\delim
    \begin{matrix}
      \text{Any } m \text{ subsequent vectors in } \\
      (e_1,\dots,e_m,a_1,\dots,a_n,e_1,\dots,e_m) \\
      \text{ are linearly independent.}
    \end{matrix} } \]
  For any two elements $X = [x_1,\dots,x_m]$ and $Y = [y_1,\dots,y_m]$
  in $\GL_m$, define the space $X_{m,n}(X,Y)$ as
  \[ X_{m,n}(X,Y) = \set{(a_1,\dots,a_n) \in (\C^m)^n
    \delim\Bigg\vert\delim
    \begin{matrix}
      \text{Any } m \text{ subsequent vectors in } \\
      (x_1,\dots,x_m,a_1,\dots,a_n,y_1,\dots,y_m) \\
      \text{ are linearly independent.}
    \end{matrix} } \]
  The special case $X=\Id$ will be denoted $\X(Y)$.
  Elements $(a_1,\dots,a_n)\in \X$ will be identified
  with the $m\times n$ matrix $A = [a_1,\dots,a_n]$ without mention.
\end{definition}

Since the space $\X(X,Y)$ only depends on linear independence of
subsequent vectors, it can be described by giving the two flags in
$\C^m$ defined from the columns of the matrices
$X=\left[x_1,\dots,x_m\right]$ and $Y=\left[y_1,\dots,y_m\right]$:
\begin{align*}
  \mathrm{Fl_R}(X) &= \Big(\spa(x_m) \subset \spa(x_{m-1},x_m) \subset
    \dots \subset \C^m\Big) \\
  \mathrm{Fl_L}(Y) &= \Big(\spa(y_1) \subset \spa(y_1,y_2) \subset
    \dots \subset \C^m\Big)
\end{align*}
Multiplying with $X^{-1}$ on each vector in $\X(X,Y)$ defines a
homeomorphism 
\[ \X(X,Y)\cong \X(\Id,X^{-1}Y) = \X(X^{-1}Y) \]
Hence we only need to consider these spaces.

Note that we can also define the space $X_{m,n}$ as the preimage of
$(\C^*)^{m+n-1}$ under the map $\Det : (\C^m)^n \to \C^{m+n-1}$,
given by
\begin{align*}
  \Det(a_1,\dots,a_n) = \Big( &\det(e_2,\dots,e_m,a_1),\\
  &\det(e_3,\dots,e_m,a_1,a_2),\\
  &\dots, \\
  &\det(a_n,e_1,\dots,e_{m-1}) \Big) 
\end{align*}
where $\det$ is the determinant map. There is a similar description
of $X_{m,n}(Y)$ as the preimage under a map $\Det_{Y}$. Since the map
$\Det_Y$ is continuous, this shows that the space $\X(Y)$ is open in
$\C^{mn}$.

\section{Stabilization}

We would like to relate the spaces $X_{m,n}$ and $X_{m,n+1}$. To do
this, consider a lower-triangular $m\times m$ invertible matrix
$L$. This preserves the right flag $\mathrm{Fl_R}(\Id)$, so it defines
a homeomorphism $\tilde L : \X(Y) \to \X(LY)$ by
\begin{align*}
  \tilde L(a_1,\dots,a_n) &= (La_1,\dots,La_n)
\end{align*}

By using the Bruhat decompositon of the general linear group,
any matrix $Y$ can be written uniquely as a product $L \sigma U$,
where $L$ is an invertible lower triangular matrix, $U$ is an
invertible upper triangular matrix, and $\sigma$ is a
permutation matrix, see \cite[Example~1.2.11]{bjorner} or
\cite[Proposition~4.5]{hiller}. Since the flag $\mathrm{Fl_L}(\sigma
U)$ is preserved by right multiplication with upper triangular
matrices, this only depends on the permutation $\sigma$ and not on
$U$. All of this together shows the following lemma:

\begin{lemma}
  For $Y\in \GL_m$ with $Y = L\sigma U$, the space $\X(Y)$ is
  homeomorphic to $\X(\sigma)$.
\end{lemma}

Using this lemma, we only need to understand the spaces
$\X(\sigma)$ to understand all of the spaces $\X(Y)$. More importantly,
it turns out that the space $X_{m,n+1}(\sigma)$ is given as a union of
spaces homeomorphic to a torus times $\X(\tau)$, for various
permutations $\tau$. This is summarized in the following lemma.

\begin{lemma}
  Any choice of indices $I= (i_1<\dots<i_k) \subset \set{1,\dots,m}$
  gives a subspace of $X_{m,n+1}(\sigma)$:
  \[ X_{m,n+1}^I(\sigma) = \set{(a_1,\dots,a_{n+1}) \in
    X_{m,n+1}(\sigma) \sdel 
    \begin{matrix} 
      (a_{n+1})_{i_j} \neq 0\; \forall i_j \in I, \\
      (a_{n+1})_j = 0 \;\forall j \not\in I
    \end{matrix} } \]
  This space is empty if $I$ does not contain the number $\sigma(m)$,
  so we will assume that $\sigma(m) \in I$ but otherwise leave it out
  of the notation.
  
  These subspaces cover $X_{m,n+1}(\sigma)$. There is a map $\varphi$,
  defined for an indexing set $I$ and a permutation $\sigma$, which
  gives a new permutation such that
  \[ X_{m,n+1}^I(\sigma) \cong \X(\varphi(I,\sigma)) \times
  (\C^*)^{\size{I}} \]
  If we define the permutation $\widehat \sigma = \sigma \cdot
  \big(m\; m-1\; \dots \; 1\big)$ and the indexing set is $I = (i_1 <
  \dots < i_k)$, then $\varphi$ is given by
  \[ \varphi\big(I,\sigma\big) = \left(\prod_{j \in J} \big(
      i_j \; \sigma(m) \big) \right) \widehat \sigma \]
  where the product is over the set
  \[ J = \set{ j \in\set{1,\dots,k} \delim\mid\delim i_j <
    \sigma(m),\; \widehat\sigma^{-1}(i_j) > \widehat\sigma^{-1}(i_r) \,
    \forall r < j } \]
\end{lemma}

The proof of this lemma relies on taking $X_{m,n+1}^I(\sigma)$ and
multiplying with a lower triangular matrix to reduce it to something
in $\X(\tau)$. By considering the matrix used, the formula
for $\tau = \varphi(I,\sigma)$ given in the lemma can be found. In
practice, it is
often easier to do the reduction by hand rather than using the lemma,
but the formula can be useful. For example, it is clear that
$\varphi(I,\sigma)$ is given as a product of transpositions
and $\varphi\big((\sigma(m)), \sigma\big)$. By
considering exactly which transpositions, we can show that for all $I$
we have
\[ \varphi(I,\sigma) \leq \varphi\big((\sigma(m)), \sigma\big) \] 
in the Bruhat ordering on the symmetric group. The Bruhat
order is defined in \cite{bjorner}. It also follows that for all
$\sigma$, 
\[ \varphi\big((1<2<\dots<m),\sigma\big) = \Id \]
With this we can identify $X_{m,n}$ with the subspace
of $X_{m,n+1}$ where the last column has a 1 on each entry:
\[ X_{m,n} \cong \set{ (a_1,\dots,a_{n+1}) \in X_{m,n+1} \mid
  a_{n+1} = (1,\dots,1) } \]
The identification is given by the map $s : \X \to X_{m,n+1}$,
\[ s(a_1,\dots,a_n) = (La_1,\dots,La_n,Le_1) \]
where $L$ is the lower triangular matrix with all entries on or below
the diagonal equal to one:
\[ L =
\begin{pmatrix}
  1 & 0 & \dots & 0 \\
  \vdots & \ddots & \ddots & \vdots \\
  \vdots &  & \ddots & 0 \\
  1 & \dots & \dots & 1
\end{pmatrix} \]

These maps give a directed system of spaces,
\[ \xymatrix{ X_{m,1} \ar[r]^s & X_{m,2} \ar[r]^s & \dots \ar[r]^s &
  X_{m,n} \ar[r]^s & X_{m,n+1} \ar[r]^s & \dots } \]
and we can take the direct limit of this system,
\[ X_{m,\infty} = \varinjlim_n \X \]
The direct limit is the disjoint union of all the spaces $\X$, with the
identification $A \sim B$ if there are $j,k$ such that
$s^k(A) = s^j(B)$. The topology is the finest such that the maps $\X
\to X_{m,\infty}$ are all continuous.

\section{Symmetries}

% Skriv noget her og byt rundt p� r�kkef�lge?
% Noget om virkninger af \C^* p� \X, e.g. virk p� s�jler, r�kker,...
There are various symmetries and group actions worth keeping in mind
when working with these spaces. Some of these will be described here.

Since we are interested in linear independence of vectors in $\C^m$,
the group of complex units $\C^*$ acts on $\X(\sigma)$ in various
ways. For any index $i$, we can define a group action $l_i$ by scaling
the $i$\textsuperscript{th} vector:
\begin{align*}
  l_i : \C^* \times \X(\sigma) &\to \X(\sigma) \\
  \big(\lambda , (a_1,\dots,a_n)\big) &\mapsto
  \big(a_1,\dots,a_{i-1},\lambda a_i,a_{i+1},\dots,a_n\big) 
\end{align*}
These actions commute with each other, so we get an action $l$ of
$(\C^*)^n$ on $\X(\sigma)$, given by
\[ l\big((\l_1,\dots,\l_n),(a_1,\dots,a_n)\big) =
\big(\l_1a_1,\dots,\l_na_n\big) \]

Likewise, we can also multiply the rows of $A\in
\X(\sigma)$ by non-zero complex numbers. This defines an action $l'$
of $(\C^*)^m$ on $\X(\sigma)$. The two actions together satisfy the
relation 
\[ l\Big((\l,\dots,\l),l'\big((\l^{-1},\dots,\l^{-1}), A\big)\Big) =
A \]
since multiplying all rows by $\l^{-1}$ while multiplying all columns
by $\l$ is the same as doing nothing.

For the special case of $\X = \X(\Id)$, there is an additional
symmetry that can be useful to know. By thinking of $A \in \X$ as a
matrix, it is possible to transpose it. Checking the various
determinants used in the definition of $\X$ show that they do not
change after transposition. Hence transposing defines a homeomorphism
$T : X_{m,n} \to X_{n,m}$, allowing us to switch the order of $m$
and $n$.

\section{The quotient space}

% Rummene Y_{m,n}. 
While the above definition of $\X$ gives an open subset of $\C^{mn}$,
it has some issues that makes the space hard to
work with. As we shall see shortly, the fundamental group and the
cohomology ring of $\X$ both become larger as $n$ grows.
This complicates things slightly, but it can be fixed by taking the
quotient of a group action.
\begin{definition}
  The space $Y_{m,n}$ is the quotient space
  \[ Y_{m,n} = \X / T^n \]
  where $T^n$ acts on $X_{m,n}$ by scaling
  the columns, as defined above:
  \[ (\lambda_1,\dots,\lambda_n)\cdot (a_1,\dots,a_n) =
  (\l_1a_1,\dots,\l_na_n) \]
\end{definition}

These spaces retain some of the structure described above. For
example, the stabilization map $s$ is linear in $A$, which means it
respects the group action.
\[ s\big((\l_1,\dots,\l_n) \cdot A\big) = (\l_1,\dots,\l_n,1)\cdot
s(A) \] 
Hence $s$ descends to a stabilization map $s : Y_{m,n} \to
Y_{m,n+1}$ and there is a direct limit,
\[ Y_{m,\infty} = \varinjlim_{n} Y_{m,n} \]
which can be thought of as $X_{m,\infty}/T^\infty$.

The space $Y_{m,n}$ is homeomorphic to the subspace of $\X$ consisting
of elements with the last $n$ coordinates of the determinant map
$\Det$ equal to 1:
\[ Y_{m,n} \cong \set{ A \in \X \delim\big\vert\delim \Det(A) \in
  (\C^*)^{m-1}\times \set{1}^n }\]
The group action gives an identification,
\[ \X \cong Y_{m,n}\times (\C^*)^n \]
This shows the claim about the fundamental group of $\X$, but it can
also be used to get information about $Y_{m,n}$ from $\X$ or
vice versa. For example, it allows us to calculate the cohomology of
one when we know the cohomology of the other, by using the
K\"unneth formula to add or remove the cohomology of $(\C^*)^n$.



%%% Local Variables: 
%%% mode: latex
%%% TeX-master: "main"
%%% End: 
 % Vis at rummene ikke er tomme (både
              % dimensions-argumentet og stabiliserings-argumentet
              % skal nok være med her.

\include{kohomologiudregning} % Bliver nok langt.... Stjæl fra noter

\include{loekker} % Introducer også \mathbb{R} og \mathbb{H}
                  % tilfældene her, og vis sætningen for disse

%\include{tre} % Specialiser til X_{3,n}-tilfældet og snak om stabilisering etc.


\backmatter
%\nocite{*}
\printbibliography{}

\end{document}
